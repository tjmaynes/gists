% File: example.tex
% Author: TJ Maynes
% This is a comment!!
\documentclass[12pt]{article}
\RequirePackage{color,graphicx}
\usepackage{newcent}
\usepackage{multicol}
\usepackage{multirow}
\usepackage{cite}
\usepackage{url}
\usepackage{xltxtra,url,parskip}
\usepackage{xunicode}
\usepackage{float}
\usepackage[ruled,lined,linesnumbered,nofillcomment]{algorithm2e}
\usepackage[absolute]{textpos}
\usepackage[left=0.7in, right=0.7in, top=0.7in, bottom=0.7in]{geometry}
\usepackage[xetex,
  unicode,
  pdfencoding=auto,
  pdfinfo={
    Title={tjmaynes/Project1},
    Author={TJ Maynes},
    Subject={TJ Maynes Project1},
    Keywords={insertion sort, quick sort, merge sort, algorithms, undergraduate},
    Producer={xelatex},
    Creator{xelatex}}]{hyperref}
\defaultfontfeatures{Scale=MatchLowercase,Mapping=tex-text}
\defaultfontfeatures{Mapping=tex-text,Scale=MatchLowercase}
\restylefloat{table}

\begin{document}

\title{Asymptotic Notation and Sorting Project}
\author{TJ Maynes}
\date{October 15, 2014}

\maketitle

\textbf{1 Theory}

\vspace{0.2in}
\textbf{1.1 Insertion Sort}

Insertion sort is a sorting method that starts at a single element in an array and then increments the rest of the elements in an array. Pseudocode for implementing an insertion sort is provided in Algorithm 1.

\begin{algorithm}
  \For{$i=1$ to $n$}{

    $j=i$

    \While{$j > 0$ and {$A[j] < A[j-1]$}}{
      $\mathtt{Swap}(A[j], A[j-1])$

      $j-=1$
    }
  }
  \caption{\texttt{Insertionsort}($A,low,high$)}
\end{algorithm}

Below is a table of CPU times (in milliseconds) from running an increasing array of various instance sizes on mergesort. \\

\begin{table}[H]
  \begin{tabular}{lllllllllll}
    & \multicolumn{10}{c}{Mergesort}                                                               \\
    & \multicolumn{10}{c}{Instance Size}                                                           \\
    & 10000  & 20000  & 30000  & 40000  & 50000  & 60000   & 70000   & 80000   & 90000   & 100000  \\
    1   & 20.736 & 36.352 & 55.552 & 81.408 & 98.816 & 118.528 & 141.568 & 159.488 & 192.256 & 206.592 \\
    2   & 17.920 & 36.608 & 63.488 & 74.496 & 93.952 & 113.664 & 142.592 & 171.264 & 185.856 & 198.912 \\
    3   & 17.408 & 36.352 & 56.320 & 81.152 & 94.464 & 114.432 & 137.728 & 169.984 & 191.488 & 208.384 \\
    4   & 17.664 & 39.680 & 59.904 & 78.592 & 94.464 & 118.016 & 142.080 & 161.536 & 178.432 & 207.360 \\
    5   & 17.408 & 36.352 & 55.552 & 75.264 & 98.048 & 115.200 & 149.760 & 165.632 & 179.456 & 212.224 \\
    Avg & 18.227 & 37.069 & 58.163 & 78.182 & 95.949 & 115.968 & 142.746 & 165.581 & 185.498 & 206.694
  \end{tabular}
  \end{table}

% this is another comment!

\end{document}
